\documentclass[conference]{IEEEtran}
\IEEEoverridecommandlockouts
% The preceding line is only needed to identify funding in the first footnote. If that is unneeded, please comment it out.
\usepackage{cite}
\usepackage{amsmath,amssymb,amsfonts}
\usepackage{algorithmic}
\usepackage{graphicx}
\usepackage{textcomp}
\usepackage{xcolor}
\usepackage{booktabs}

\usepackage{subfigure}
\usepackage{hyperref}
\usepackage{makecell}
\usepackage{adjustbox}

\newcommand{\note}[1]{{\color{blue} #1}}
\newcommand{\ZZ}[1]{{\color{red} #1}}

% \newcommand{\note}{\underline}

\hypersetup{hidelinks}
% \hypersetup{draft}

\def\BibTeX{{\rm B\kern-.05em{\sc i\kern-.025em b}\kern-.08em
    T\kern-.1667em\lower.7ex\hbox{E}\kern-.125emX}}
\begin{document}

% \title{Conference Paper Title*\\
% {\footnotesize \textsuperscript{*}Note: Sub-titles are not captured in Xplore and
% should not be used}
% \thanks{Identify applicable funding agency here. If none, delete this.}
% }

\title{VQA programs Synthesis by Clifford Formalism}



% \author{
%     \IEEEauthorblockN{Zhaohui Yang}
%     \IEEEauthorblockA{\textit{Department of Electrical \& Computer Engineering} \\
%     \textit{University of Arizona}\\
%     Tucson, AZ, USA\\
%     zhy@arizona.edu}
% }


\author{
    \IEEEauthorblockN{Zhaohui Yang\IEEEauthorrefmark{1}, David Ding\IEEEauthorrefmark{2}, Jianxin Chen\IEEEauthorrefmark{3}, Yuan Xie\IEEEauthorrefmark{1}}
    
    % \IEEEauthorblockN{Zhaohui Yang\IEEEauthorrefmark{1}, Chaohan Cui\IEEEauthorrefmark{2}, Zheshen Zhang\IEEEauthorrefmark{3}\IEEEauthorrefmark{2}\IEEEauthorrefmark{1}\IEEEauthorrefmark{4}}
    
    \IEEEauthorblockA{\textit{\IEEEauthorrefmark{1}Department of Electronic and Computer Engineering, The Hong Kong University of Science and Technology, Hong Kong}}

    \IEEEauthorblockA{\textit{\IEEEauthorrefmark{2} Yau Mathematical Sciences Center, Tsinghua University, Beijing, China}}

    \IEEEauthorblockA{\textit{\IEEEauthorrefmark{3}Department of Computer Science and Technology, Tsinghua University, Beijing, China}}


    % \IEEEauthorblockA{\{zhaohui\}@ucsb.edu, }
}


\maketitle


\begin{abstract}
    Quantum computing ...

    PHOENIX: Pauli-based High-level Optimization ENgine for Instruction eXecution on NISQ-Era quantum devices

    
    
\end{abstract}

% \begin{IEEEkeywords}
% Entanglement distribution, network protocol design, platform-as-a-service, quantum communication, quantum internet service provider, quantum network, reconfigurable network architecture, routing, software-defined network.
% \end{IEEEkeywords}



%%%%%%%%%%%%%%%%%%%%%%%%%%%%%%%%%%%%%%%%%%%%%%%%
% Introduction
%%%%%%%%%%%%%%%%%%%%%%%%%%%%%%%%%%%%%%%%%%%%%%%%

\section{Introduction}

    Quantum computing ...
    

    % Some recent demonstrative experiments of entanglement distribution have begun to use CV entangled photons \cite{zhang2008distribution}, especially for long-distance scenarios.


    % The rest of this paper is structured as follows. In Sec. \ref{sec:2:resutls}, we firstly describe the hardware infrastructures and their re-constructing approaches on basis of the existing fiber-based network at the University of Arizona (UA), and then explain how the QISP is implemented under a reconfigurable framework. Field-test evaluation results are also included in this section. We summarize our work in Sec. \ref{sec:3:discussion} with analyses of current limitations and future work directions. In Sec. \ref{sec:4:methods}, we will see in more detail the Quagent's architecture and the entanglement source characterization methods.
    




%%%%%%%%%%%%%%%%%%%%%%%%%%%%%%%%%%%%%%%%%%%%%%%%%%%%%%%%%
% Acknowledgement
%%%%%%%%%%%%%%%%%%%%%%%%%%%%%%%%%%%%%%%%%%%%%%%%%%%%%%%%%
% \section*{Acknowledgement}
% ...

%%%%%%%%%%%%%%%%%%%%%%%%%%%%%%%%%%%%%%%%%%%%%%%%%%%%%%%%%
% Reference
%%%%%%%%%%%%%%%%%%%%%%%%%%%%%%%%%%%%%%%%%%%%%%%%%%%%%%%%%

% \bibliographystyle{IEEEtran}
% \bibliography{reference}


%%%%%%%%%%%%%%%%%%%%%%%%%%%%%%%%%%%%%%%%%%%%%%%%%%%%%%%%%
% Appendix
%%%%%%%%%%%%%%%%%%%%%%%%%%%%%%%%%%%%%%%%%%%%%%%%%%%%%%%%%


% \appendix{...}




\end{document}

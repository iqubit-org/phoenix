\documentclass[conference]{IEEEtran}
\IEEEoverridecommandlockouts
% The preceding line is only needed to identify funding in the first footnote. If that is unneeded, please comment it out.
\usepackage{cite}
\usepackage{amsmath,amssymb,amsfonts}
% \usepackage{algorithmic}
\usepackage[ruled, vlined, linesnumbered]{algorithm2e}
\usepackage{mhchem}
\usepackage{graphicx}
\usepackage{textcomp}
\usepackage{xcolor}
\usepackage{booktabs}

\usepackage{subfigure}
\usepackage{hyperref}
\usepackage{makecell}
\usepackage{adjustbox}

\newcommand{\note}[1]{{\color{blue} #1}}
\newcommand{\ZY}[1]{{\color{purple}[ZY: #1]}}

\newcommand{\phoenix}{\textsc{Phoenix}}
\newcommand{\qiskit}{\textsc{Qiskit}}
\newcommand{\tket}{\textsc{TKet}}
\newcommand{\tetris}{\textsc{Tetris}}
\newcommand{\paulihedral}{\textsc{Paulihedral}}
\newcommand{\pcaost}{\textsc{PCOAST}}
\newcommand{\twoqan}{\textsc{2QAN}}


\SetAlFnt{\small}
% \SetAlCapFnt{\small}



% \newcommand{\note}{\underline}

\hypersetup{hidelinks}
% \hypersetup{draft}

\def\BibTeX{{\rm B\kern-.05em{\sc i\kern-.025em b}\kern-.08em
    T\kern-.1667em\lower.7ex\hbox{E}\kern-.125emX}}
\begin{document}

% \title{Conference Paper Title*\\
% {\footnotesize \textsuperscript{*}Note: Sub-titles are not captured in Xplore and
% should not be used}
% \thanks{Identify applicable funding agency here. If none, delete this.}
% }

\title{\phoenix: Pauli-based High-level Optimization Engine for Instruction Execution on NISQ Devices}

% VQA Programs Synthesis by Clifford Formalism



% \author{
%     \IEEEauthorblockN{Zhaohui Yang}
%     \IEEEauthorblockA{\textit{Department of Electrical \& Computer Engineering} \\
%     \textit{University of Arizona}\\
%     Tucson, AZ, USA\\
%     zhy@arizona.edu}
% }


% \author{
%     \IEEEauthorblockN{Zhaohui Yang\IEEEauthorrefmark{1}, David Ding\IEEEauthorrefmark{2}, Jianxin Chen\IEEEauthorrefmark{3}, Yuan Xie\IEEEauthorrefmark{1}}
    
%     \IEEEauthorblockA{\textit{\IEEEauthorrefmark{1}Department of Electronic and Computer Engineering, The Hong Kong University of Science and Technology, Hong Kong}}

%     \IEEEauthorblockA{\textit{\IEEEauthorrefmark{2} Yau Mathematical Sciences Center, Tsinghua University, Beijing, China}}

%     \IEEEauthorblockA{\textit{\IEEEauthorrefmark{3}Department of Computer Science and Technology, Tsinghua University, Beijing, China}}


%     % \IEEEauthorblockA{\{zhaohui\}@ucsb.edu, }
% }


\maketitle


\begin{abstract}
    Variational quantum algorithms based on Hamilton simulation represent a specialized class of quantum programs well-suited for near-term quantum computing applications due to its modest resource requirements in terms of qubits and circuit depth. Unlike the conventional single-qubit/two-qubit gate sequence representation, Hamiltonian simulation programs are composed of disciplined subroutines known as Pauli exponentiations (Pauli strings with coefficients) that are variably arranged. To capitalize on these distinct program features, this study introduces \phoenix, a highly effective compilation framework that primarily operates at the high-level Pauli-based intermediate representation (IR) for generic Hamiltonian simulation programs. \phoenix\ exploits global program optimization opportunities to the largest extent, compared to existing SOTA methods despite some of them also utilizing similar IRs. \phoenix\ employs the binary symplectic form (BSF) to formally describe Pauli strings and reformulates IR synthesis as a sequence of Clifford transformations on BSF. It comes with a heuristic BSF simplification algorithm that simultaneously reduces Pauli strings' weights to two-qubit operations by identifying the most appropriate Clifford operators. \phoenix\ further performs a global ordering strategy in a Tetris-like fashion for these simplified IR groups, carefully balancing optimization opportunities for gate cancellation, minimizing circuit depth, and managing qubit routing overhead. Overall, \phoenix\ outperforms other SOTA dedicated compilers in benchmarking across diverse program categories, backend ISAs, and hardware topologies.
    
    
\end{abstract}

% \begin{IEEEkeywords}
% Entanglement distribution, network protocol design, platform-as-a-service, quantum communication, quantum internet service provider, quantum network, reconfigurable network architecture, routing, software-defined network.
% \end{IEEEkeywords}



%%%%%%%%%%%%%%%%%%%%%%%%%%%%%%%%%%%%%%%%%%%%%%%%
% Introduction
%%%%%%%%%%%%%%%%%%%%%%%%%%%%%%%%%%%%%%%%%%%%%%%%

\section{Introduction}

    Quantum computing ...
    

    % Some recent demonstrative experiments of entanglement distribution have begun to use CV entangled photons \cite{zhang2008distribution}, especially for long-distance scenarios.


    % The rest of this paper is structured as follows. In Sec. \ref{sec:2:resutls}, we firstly describe the hardware infrastructures and their re-constructing approaches on basis of the existing fiber-based network at the University of Arizona (UA), and then explain how the QISP is implemented under a reconfigurable framework. Field-test evaluation results are also included in this section. We summarize our work in Sec. \ref{sec:3:discussion} with analyses of current limitations and future work directions. In Sec. \ref{sec:4:methods}, we will see in more detail the Quagent's architecture and the entanglement source characterization methods.



%%%%%%%%%%%%%%%%%%%%%%%%%%%%%%%%%%%%%%%%%%%%%%%%
% Motivation
%%%%%%%%%%%%%%%%%%%%%%%%%%%%%%%%%%%%%%%%%%%%%%%%

\section{Motivation}

\ZY{Motivation and preliminary knowledge}


\section{Our Propsal: \phoenix}

\subsection{Overall framework}

\subsection{BSF simplification for each IR group}


    \begin{algorithm}[tbp]
        \SetAlgoLined
        \caption{Pauli Strings Simplification in BSF}
        \label{algo:simplification}
        \SetKwInOut{Input}{Input}
        \SetKwInOut{Output}{Output}
        \SetKwBlock{Assumption}{Assumption}{}
    
        % \Input{Pauli strings with corresponding coefficients (\textit{pls}, \textit{coes})}
        \Input{Pauli strings list \textit{pls}}
        \Output{Reconfigured circuit components list \textit{cfg}}
    
        % \tcp{Following pseudocode only involves transformation on pls while omitting coes WLOG}
        \BlankLine
        \textit{cfg} $ \gets \emptyset $;\quad
        \textit{bsf} $ \gets \textsc{BSF}$(\textit{pls});\quad
        \textit{cliffs\_with\_locals} $\gets \emptyset$\; %\tcp*{Cliffords with local Paulis}
        % $ n \gets $ \textit{bsf}.\textsc{numQubits}()\;
        \While{bsf.\textsc{totalWeight()} $>$ 2}{
            \textit{local\_bsf} $\gets$ \textit{bsf}.\textsc{popLocalPaulis}()\; 
            % $ C \gets \emptyset$;\quad
            % $ B \gets \emptyset$;\quad
    
            $ C \gets \emptyset$ \tcp*{Clifford2Q candidates}
            $ B \gets \emptyset$ \tcp*{Each element of $B$ results from applying each Clifford2Q candidate on \textit{bsf}}
            \textit{costs} $\gets \emptyset$ \tcp*{Cost functions calculated on each element of $B$}
            \For{cg \textbf{in} \textsc{CLIFFORD\_2Q\_SET}}{
                \For{i, j \textbf{in} $ \textsc{combinations}(\textsc{range}(n), 2) $}{
                    \textit{cliff} $\gets$ \textit{cg}.\textsc{on}$ (i, j) $ \tcp*{qubits acted on}
                    \textit{bsf}$'$ $\gets$ \textit{bsf}.\textsc{applyClifford2Q}(\textit{cliff})\;
                    \textit{cost} $\gets$ \textsc{calculateBSFCost}(\textit{bsf}$'$)\;
                    $ C.\textsc{append}$(\textit{cliff})\;
                    $ B.\textsc{append}$(\textit{bsf}$'$)\;
                    \textit{costs}.\textsc{append}(\textit{cost})\;
                }
            }
            % \textit{bsf} $\gets$ \textit{B}[\textit{costs}.\textsc{index}(\textit{min}(\textit{costs}))]\;
            % \textit{cliff} $\gets$ \textit{C}[\textit{costs}.\textsc{index}(\textit{min}(\textit{costs}))]\;
            \textit{bsf} $ \gets \textsc{BSFWithMinCost} (B, \textit{costs}) $\;
            \textit{cliff} $ \gets \textsc{CliffordWithMinCost} (C, \textit{costs}) $\;
            \textit{cliffs\_with\_locals}.\textsc{append}((\textit{cliff}, \textit{local\_bsf}))\;
        }
        \BlankLine
        \textit{cfg}.\textsc{append}(\textit{bsf})\;
        \For{cliff, local\_bsf \textbf{in} cliffs\_with\_locals}{
            \tcp{Clifford2Q operators are added as conjugations, with local Pauli strings peeled before each epoch}
            \textit{cfg}.\textsc{prepend}(\textit{cliff})\;
            \textit{cfg}.\textsc{append}(\textit{local\_bsf})\;
            \textit{cfg}.\textsc{append}(\textit{cliff})\;
        }
    
    \end{algorithm}


\subsection{Ordering of IR groups}


%%%%%%%%%%%%%%%%%%%%%%%%%%%%%%%%%%%%%%%%%%%%%%%%%%%%%%%%%
% Evaluation
%%%%%%%%%%%%%%%%%%%%%%%%%%%%%%%%%%%%%%%%%%%%%%%%%%%%%%%%%

\section{Evaluation}


\subsection{Experimental settings}


\begin{table}[tbp]
    \centering
    \caption{UCCSD benchmark suite.}
    \setlength{\tabcolsep}{4.2pt}
    % \fontsize{5}{5}\selectfont
    \scalebox{0.876}{
        \begin{tabular}{|l|r|r|r|r|r|r|r|}
    % \toprule
    \hline
    \textbf{Benchmark} & \textbf{\#Qubit} & \textbf{\#Pauli} & $\mathbf{w_{\max}}$ & \textbf{\#Gate} & \textbf{\#CNOT}  & \textbf{Depth} & \textbf{Depth-2Q} \\
    % \midrule
    \hline
    CH2\_cmplt\_BK & 14 & 1488 & 10 & 37780 & 19574 & 23568 & 19399 \\
    \hline
    CH2\_cmplt\_JW & 14 & 1488 & 14 & 34280 & 21072 & 23700 & 19749 \\
    \hline
    CH2\_frz\_BK & 12 & 828 & 10 & 19880 & 10228 & 12559 & 10174 \\
    \hline
    CH2\_frz\_JW & 12 & 828 & 12 & 17658 & 10344 & 11914 & 9706 \\
    \hline
    H2O\_cmplt\_BK & 14 & 1000 & 10 & 25238 & 13108 & 15797 & 12976 \\
    \hline
    H2O\_cmplt\_JW & 14 & 1000 & 14 & 23210 & 14360 & 16264 & 13576 \\
    \hline
    H2O\_frz\_BK & 12 & 640 & 10 & 15624 & 8004 & 9691 & 7934 \\
    \hline
    H2O\_frz\_JW & 12 & 640 & 12 & 13704 & 8064 & 9332 & 7613 \\
    \hline
    LiH\_cmplt\_BK & 12 & 640 & 10 & 16762 & 8680 & 10509 & 8637 \\
    \hline
    LiH\_cmplt\_JW & 12 & 640 & 12 & 13700 & 8064 & 9342 & 7616 \\
    \hline
    LiH\_frz\_BK & 10 & 144 & 9 & 2890 & 1442 & 1868 & 1438 \\
    \hline
    LiH\_frz\_JW & 10 & 144 & 10 & 2850 & 1616 & 1985 & 1576 \\
    \hline
    NH\_cmplt\_BK & 12 & 640 & 10 & 15624 & 8004 & 9691 & 7934 \\
    \hline
    NH\_cmplt\_JW & 12 & 640 & 12 & 13704 & 8064 & 9332 & 7613 \\
    \hline
    NH\_frz\_BK & 10 & 360 & 9 & 8303 & 4178 & 5214 & 4160 \\
    \hline
    NH\_frz\_JW & 10 & 360 & 10 & 7046 & 3896 & 4640 & 3674 \\
    % \bottomrule
    \hline
\end{tabular}    

    }
    \label{tab:uccsd}
    
\end{table}

\subsection{Benchmarks}


\subsubsection{Metrics}

\subsubsection{Baselines}

\begin{itemize}
    \item \tket
    \item \paulihedral
    \item \tetris
\end{itemize}



\subsection{Logical-level compilation}


\begin{figure}[tbp]
    \centering
    \includegraphics[width=\columnwidth]{figures/all2all.pdf}
    \caption{Bechmarking on logical-level synthesis (all2all topology)}
    \label{fig:all2all}
\end{figure}


\subsection{Breakdown analysis}


\subsection{QAOA benchmarking}



\begin{figure}[tbp]
    \centering
    \includegraphics[width=\columnwidth]{figures/qaoa.pdf}
    \caption{QAOA benchmarking}
    \label{fig:qaoa}
\end{figure}



\begin{table}[btp]
    \centering
    \caption{QAOA benchmarking versus 2QAN.}
    \setlength{\tabcolsep}{3.8pt}
    \scalebox{0.78}{
        \begin{tabular}{lrrrrrrrrrr}
    \toprule
    fname & num_qubits & num_paulis & num_cnot(manhattan)(2qan) & num_cnot(manhattan)(phoenix) & depth_2q(manhattan)(2qan) & depth_2q(manhattan)(phoenix) & num_swap(manhatten)(2qan) & num_swap(manhatten)(phoenix) & routing_overhead(2qan) & routing_overhead(phoenix) \\
    \midrule
    qaoa_rand_16 & 16 & 32 & 168 & 150 & 85 & 52 & 37 & 29 & 2.62 & 2.34 \\
    qaoa_rand_20 & 20 & 40 & 217 & 187 & 85 & 49 & 47 & 39 & 2.71 & 2.34 \\
    qaoa_rand_24 & 24 & 48 & 274 & 257 & 100 & 67 & 63 & 56 & 2.85 & 2.68 \\
    qaoa_reg3_16 & 16 & 24 & 149 & 99 & 61 & 28 & 44 & 17 & 3.10 & 2.06 \\
    qaoa_reg3_20 & 20 & 30 & 172 & 128 & 46 & 30 & 46 & 23 & 2.87 & 2.13 \\
    qaoa_reg3_24 & 24 & 36 & 218 & 158 & 62 & 34 & 62 & 30 & 3.03 & 2.19 \\
    \bottomrule
\end{tabular}
    
    }
    \label{tab:qaoa}
\end{table}


\subsection{Hardware-aware compilation}


\begin{figure}[tbp]
    \centering
    \includegraphics[width=\columnwidth]{figures/num_2q_gates_manhattan.pdf}
    \caption{Hardware-aware compilation for limited-topology NISQ device}
    \label{fig:manhattan}
\end{figure}


\begin{figure}[tbp]
    \centering
    \includegraphics[width=\columnwidth]{figures/algo_err.pdf}
    \caption{Algorithmic error}
    \label{fig:algo-err}
\end{figure}




\subsection{Diverse ISA comparison}


\begin{figure}[tbp]
    \centering
    \includegraphics[width=\columnwidth]{figures/su4_comparison.pdf}
    \caption{SU(4) ISA comparison}
    \label{fig:su4-isa}
\end{figure}






\subsection{Real system evaluation}


\subsection{Scalability}







%%%%%%%%%%%%%%%%%%%%%%%%%%%%%%%%%%%%%%%%%%%%%%%%%%%%%%%%%
% Acknowledgement
%%%%%%%%%%%%%%%%%%%%%%%%%%%%%%%%%%%%%%%%%%%%%%%%%%%%%%%%%
% \section*{Acknowledgement}
% ...

%%%%%%%%%%%%%%%%%%%%%%%%%%%%%%%%%%%%%%%%%%%%%%%%%%%%%%%%%
% Reference
%%%%%%%%%%%%%%%%%%%%%%%%%%%%%%%%%%%%%%%%%%%%%%%%%%%%%%%%%

% \bibliographystyle{IEEEtran}
% \bibliography{reference}


%%%%%%%%%%%%%%%%%%%%%%%%%%%%%%%%%%%%%%%%%%%%%%%%%%%%%%%%%
% Appendix
%%%%%%%%%%%%%%%%%%%%%%%%%%%%%%%%%%%%%%%%%%%%%%%%%%%%%%%%%


% \appendix{...}




\end{document}
